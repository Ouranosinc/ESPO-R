\documentclass[letterpaper,10pt]{article}
\usepackage{amsmath}
\usepackage{hyperref}

\begin{document}
\title{Bias Adjustment of ESPO-R5 v1.0}
\author{Pascal Bourgault, Travis Logan, Trevor J. Smith, Juliette Lavoie}
\maketitle

The temperature and precipitation data from the simulations were first extracted over an area covering North America and, if necessary, converted into daily values.
Then using the \href{https://earthsystemmodeling.org/regrid/}{ESMF software}, accessed through its python \href{https://xesmf.readthedocs.io/en/latest/}{xESMF} interface, all the extracted simulation data is interpolated 
bilinearly to the ERA5-Land grid.

The ESPO-R5 v1.0 bias adjustment procedure then uses \href{https://xclim.readthedocs.io/en/stable/sdba.html}{xclim} algorithms to adjust simulation bias following a quantile mapping procedure.
In particular, the algorithm used is inspired by the "Detrended Quantile Mapping" method described by Cannon (2015).
The procedure is univariate, bipartite, acts differently on the trends and the anomalies and is applied iteratively on each day of the year (grouping) and on each grid point.

\subsection{Data}
The adjustment target is the reference $Y_r$ and the simulated calibration data is $X_c$, both extracted over the 1981-2010 reference period.
The simulation data to be adjusted is $X_s$, it includes all timesteps (1951-2100).
The bias-adjusted data (or "scenario") is $X_{ba}$.

\subsection{Grouping}
Data is adjusted for each day of the year, using a rolling window of 31 days.
For example, the adjustment for February 1 (day 32) is calibrated using data from January 15 to February 15, over the 30 years of the reference period. 
During the adjustment itself, the adjustment is used for February 1st only, for all years of the simulation. 

\subsection{Detrending}
For each day of the year and each grid points, we first compute the averages and anomalies of the reference data ($\overline{Y_r}, Y_r'$) and the simulations ($\overline{X_h}, X_h'$) over the reference period, 1981-2010.

Instead of a simple moving mean, $X_s$ is detrended with a locally weighted regression (LOESS; Cleveland, 1979).
We chose this method for its slightly heavier weights given at the center of the moving window, reducing impacts of abrupt interannual changes on the trend and anomalies. It also has a more robust handling of the extremeties of the timeseries. The LOESS window had a 30-year width and a tricube shape, the local regression was of degree 0 and only one iteration was performed. The detrending was applied on each day of the year but after averaging over the 31 day window and it yielded the trend $\overline{X_s}$ and the residuals $X_s'$.

\subsection{Algorithm}
\subsubsection{Adjustment of the residuals}
With $F_{Y_r'}$ and $F_{X_h'}$ the empirical cumulative distribution functions (CDF) of $Y_r'$ and $X_r'$ respectively, the adjustment of anomalies is split in two steps. In the first, an adjustment factor function is computed:

\begin{align*}


\begin{align*}
X_{ba}' &= X_s' + F^{-1}_{Y_r'}\left(F_{X_c'}\left(X_s'\right)\right) - F^{-1}_{X_c'}\left(F_{X_c'}\left(X_s'\right)\right) \tag{additive} \\
X_{ba}' &= X_s'\frac{F^{-1}_{Y_r'}\left(F_{X_c'}\left(X_s'\right)\right)}{F^{-1}_{X_c'}\left(F_{X_c'}\left(X_s'\right)\right)} \tag{multiplicative}
\end{align*}

The implemented algorithm splits the process in two steps. doesn't use the full CDF

\subsubsection{Adjustment of the trend}

An offset factor is computed betwen the averages.
The entire simulation is detrended using a Loess with a width of 30 years.
The residuals corrected with the adjustment factors.
The trend is adjusted with the offset factor.

\subsection{Variables}
Adjustments are applied separately for each of the 3 variables.
Adjusting \texttt{tasmax} and  \texttt{tasmin} independantly can lead to physical inconsistencies (\texttt{tasmin} > \texttt{tasmax}) in the final data (Thrasher et al., 2012, Agbazo and Grenier, 2020).
Instead, we compute the daily temperature range (or amplitude \texttt{dtr = tasmax - tasmax}) and adjust this variable in addition to \texttt{tasmax} and  \texttt{pr}.


\subsection{Pre-processing of precipitation}
Our quantile mapping methods are prone to 


\texttt{tasmax} is adjusted in an additive way : the adjustment and offset factors are added to the residuals and trend.
\texttt{pr} and \texttt{dtr} are adjusted in an multiplicative way : the adjustment and offset (scaling) factors are multiplied to the residuals and trend.
Although computational more expensive the rolling window method allows for better adjustment of the annual cycle.
Note, this method does not work well with leap years as there is 4 times less data for day 366.
To remedy this problem, all simulations as well as the reference product are converted to this "noleap" calendar. 

\end{document}